\documentclass[aspectratio=169]{beamer}

%\usepackage[T1]{fontenc}

\usepackage{listings}
\usepackage{graphicx}
\usepackage{lmodern}
\usepackage{colonequals}
\usepackage{amsmath}
\usepackage[spanish]{babel}
\usepackage{hyperref}
\usepackage{color}
\usepackage{multicol}
\usepackage{fontspec}
\setmonofont{Fira Code}[
     Contextuals = Alternate,
     Ligatures = TeX,
]
\newsavebox{\mypostbreak}
\savebox{\mypostbreak}{\mbox{\textcolor{red}{$\hookrightarrow$}\space}}

\lstset{
  basicstyle=\ttfamily,
  extendedchars=true,
  inputencoding=utf8,
  captionpos=b,
  breaklines=true,
  columns=fullflexible,
  postbreak=\usebox{\mypostbreak},
  escapeinside={(*@}{@*)},
}


\title{Similitud Coseno}

\author{Iván Molina Rebolledo}

\institute{Benemérita Universidad Autónoma de Puebla}
\date{Primavera 2022\\\today}

\begin{document}

\frame{\titlepage}

\begin{frame}
\frametitle{Motivación}
En el analisís de documentos, la comparación de los mismos es muy importante
para extraer información relevante que puede ser usada en consultas. o
extracción de datos. \pause

Por eso resulta necesario buscar maneras de poder comparar documentos, y
que nos permita tener un valor de similitud preciso para poder describir
a detalle los documentos que se comparan.
\end{frame}

\begin{frame}
\frametitle{Similitud Coseno}
Una de las formas de comparar documentos es la similitud coseno.
La similitud coseno es una medida de similitud entre dos documentos,
que se calcula a partir de la suma de productos de los vectores
de pesos de cada uno de los documentos. \pause

Esto es dividido entre el producto de la raiz de la suma de los cuadrados
de cada uno de los elementos de los vectores de pesos. \pause

Más adelante veremos a detalle cómo se implementa esta similitud.
\end{frame}

\begin{frame}
\frametitle{Diseño del proyecto}
Este proyecto está dividido en diversas etapas internas que nos permiten
tener un control total de lo que se está haciendo en cada paso. \pause

Estas etapas se explican a continuación en las siguientes diapositivas.
\end{frame}

\begin{frame}
\frametitle{Parsing}
En esta etapa se realiza el parsing de los documentos, y se obtiene una
lista de los documentos que se van a analizar. \pause

Aquí es también donde los documentos son procesados. Pasamos cada uno
de los textos que tokenizan y elimnan simbolos innecesarios. \pause

Como último paso se realiza el stemming de los documentos. \pause
Esto es hecho a partir de una librería de steaming programada en
C. \pause Sin embargo, es un poco antigua, así que puede no ser tan
eficiente como sus contrapartes en lenguajes como Python.
\end{frame}

\begin{frame}[fragile]
\frametitle{Stage 0 (Estructura de datos)}
Es importante definir una estructura de datos para los documentos.
Nosotros la establecimos como se muestra a continuación: \pause

\lstinputlisting[firstline=13, lastline=19]{src/IndexRep.hs}

Además en esta etapa se incializa la estructura de datos. \pause
Establemos las palabras y el número de documentos, mientras que
lo demás lo incializamos vacío.
\end{frame}

\begin{frame}
\frametitle{Stage 1}
En esta etapa no sucede mucho. \pause
Sólo se realiza el proceso de conteo de palabras y se guarda en la
estructura de datos. \pause

Al finalizar, tendremos una estructura de datos actualizada.
\end{frame}

\begin{frame}[fragile, allowframebreaks]
\frametitle{Stage 2}
Esta etapa se encarga de calcular los pesos para cada uno de los
terminos. Esto está dado por la siguientes funciones:

\lstinputlisting[firstline=91, lastline=96]{src/IndexRep.hs}

\framebreak

\lstinputlisting[firstline=91, lastline=96]{src/IndexRep.hs}

\framebreak

\lstinputlisting[firstline=98, lastline=100]{src/IndexRep.hs}

\framebreak

Basta con proveer de los indices adecuados (así como de la
estructura de datos) para que se pueda realizar el cálculo de los
pesos.

\end{frame}

\begin{frame}[fragile]
\frametitle{Similitud Coseno}
Está definción es el centro de nuestro proyecto. \pause

\lstinputlisting[firstline=74, lastline=78]{src/Glue.hs}

\framebreak

Sin embargo la función no es invocable directamente, sino que se
debe invocar desde una función externa que se encarga de realizar
el cálculo de la similitud coseno para todos los pares de documentos.

\end{frame}


\begin{frame}
\frametitle{Ensamblaje (\textit{main})]}
En esta etapa se realiza el ensamblaje del proyecto. \pause
Definimos cada una de las funciones que nos permiten generar las
tablas de datos. \pause

Las tablas generadas son exportadas a archivos \LaTeX{}, para que
puedan ser visualizadas en el documento principal de este proyecto.
\end{frame}

\begin{frame}
\frametitle{Datos}
Por desgracia, no se puede visualizar todos los datos que se generan
en una sola diapositiva. \pause

Tan sólo una de las tablas tiene más de cuatrocientosmil registros. \pause

Por suerte \LaTeX{} es capaz de visualizar una tabla de datos tan grande,
y es la manera en la que se muestran los datos en el documento.
\end{frame}

\begin{frame}
\frametitle{Conclusiones}
El proceso de indexación es lento y puede ser demorado. \pause
Esto se debe a la forma que manejamos los datos para ser procesados.\pause

Aparte Haskell es un lenguaje inmutable, por lo que el costo de
modificar una estructura de datos es muy grande. \pause
Sin embargo, existen alternativas que pueden ser muy eficientes. \pause

En cuanto a la similitud coseno, calcular todos los pares de documentos
es una tarea muy grande. \pause De hecho, es el paso más lento y
complicado de computar. \pause Aunque esto se puede deber a la
cantidad de documentos que estamos tratando. \pause

Sin embargo los tiempos de ejecución mejoran al compilarse a código
nativo con GHC. \pause Mientras que el cálculo con GHCI es mucho más
lento.
\end{frame}

\begin{frame}
\frametitle{Muchas gracias}
Este ha sido un proyecto de investigación de la asignatura de
Recuperación de Información.

«Espacio para dudas y comentarios, demostración del código»
\end{frame}

\end{document}
